\documentclass[12pt]{article}
 
\usepackage[margin=1in]{geometry} 
\usepackage{amsmath,amsthm,amssymb}
\usepackage[margin=1in]{geometry} 
\usepackage{mathrsfs}
\usepackage{amsmath,amsthm,amssymb}
%Para hacer dos columnas en latex
\usepackage{multicol} 
\usepackage[spanish]{babel} %Castellanización
\usepackage[T1]{fontenc} %escribe lo del teclado
\usepackage[utf8]{inputenc} %Reconoce algunos símbolos
\usepackage{lmodern} %optimiza algunas fuentes
\usepackage{graphicx}
\graphicspath{ {images/} }
\usepackage{hyperref} % Uso de links
\usepackage[spanish]{babel} % para escribir en espanol
%\usepackage[latin1]{inputenc} % para acentos sin codigo
\usepackage{multirow, array} % para las tablas
\usepackage{float} % para usar [H]
\usepackage{vmargin}
\usepackage{pdflscape} % para poner horizontal alguna hoja
\usepackage{tcolorbox}
%Este paquete tiene el salto de pagina
\usepackage{afterpage}


\setpapersize{A4}
\setmargins{1.5cm}     % margen izquierdo
{1cm}              % margen superior
{18cm}               % anchura del texto
{26.5cm}               % altura del texto
{1pt}                % altura de los encabezados
{1cm}                % espacio entre el texto y los encabezados
{0pt}                % altura del pie de página
{0.3cm}               % espacio entre el texto y el pie de página

\newcommand{\N}{\mathscr{N}}
\newcommand{\Z}{\mathscr{Z}}
 \newcommand{\X}{\mathscr{X}}
 
\newenvironment{theorem}[2][Theorem]{\begin{trivlist}
\item[\hskip \labelsep {\bfseries #1}\hskip \labelsep {\bfseries #2.}]}{\end{trivlist}}
\newenvironment{lemma}[2][Lemma]{\begin{trivlist}
\item[\hskip \labelsep {\bfseries #1}\hskip \labelsep {\bfseries #2.}]}{\end{trivlist}}
\newenvironment{exercise}[2][Exercise]{\begin{trivlist}
\item[\hskip \labelsep {\bfseries #1}\hskip \labelsep {\bfseries #2.}]}{\end{trivlist}}
\newenvironment{problem}[2][Problem]{\begin{trivlist}
\item[\hskip \labelsep {\bfseries #1}\hskip \labelsep {\bfseries #2.}]}{\end{trivlist}}
\newenvironment{question}[2][Question]{\begin{trivlist}
\item[\hskip \labelsep {\bfseries #1}\hskip \labelsep {\bfseries #2.}]}{\end{trivlist}}
\newenvironment{corollary}[2][Corollary]{\begin{trivlist}
\item[\hskip \labelsep {\bfseries #1}\hskip \labelsep {\bfseries #2.}]}{\end{trivlist}}

\newenvironment{solution}{\begin{proof}[Solution]}{\end{proof}}
 
 
 
 
 
 
 
 
 
\begin{document}
 
%\begin{landscape}  %esto es para poner en horizontal esta pagina


\begin{center}
  \Huge{Material de Apoyo}
\end{center}

\begin{center}
\Large{Inferencia Estadística No Parametrica}  \\  
\small{Yhon Paúl Tiahuallpa Yucra}\\
\end{center}


\section{Estadísticos de Orden}

\begin{tcolorbox}[colback=white,colframe=black,width=12cm ,coltitle=white, title = Función de Densidad $X_{(j)}$  ]
\begin{center}
$
f_{X_{(j)}}(x) = \frac{n!}{(j-1)!(n-j)!} f_{X}(x)\left [ F_X (x) \right ]^{j-1} \left [ 1 -  F_X (x) \right ]^{n-j}
$
\end{center}
\end{tcolorbox}



\begin{tcolorbox}[colback=white,colframe=black,width=18cm ,coltitle=white, title = Función de Distribución Conjunta $X_{(i)}$ y $X_{(j)}$  ]
\begin{center}
$
\footnotesize{
f_{X_{(i)},X_{(j)}}(u, \upsilon) = \frac{n!}{(i-1)!(j-1-i)!(n-j)!} f_{X}(u)f_{X}(\upsilon)\left [ F_X (u) \right ]^{i-1} \left [ F_{X}(\upsilon) -  F_X (u) \right ]^{j-1-i} \left [ 1 -  F_X (\upsilon) \right ]^{n-j}
}  $
\end{center}
\end{tcolorbox}



\begin{tcolorbox}[colback=white,colframe=black,width=12cm ,coltitle=white , title = Densidad conjunta del máximo y el mínimo]
\begin{center}
$
f_{X_{1:n},X_{n:n}}(x, y) = n(n-1)\left [ F(y) - F(x) \right ]^{n-2} f(x)f(y) \quad ,x<y
$ 
\end{center}
\end{tcolorbox}




\begin{tcolorbox}[colback=white,colframe=black,width=17cm ,coltitle=white, title = Densidad conjunta de dos estadísticas de orden consecutivas]
\begin{center}
$
f_{X_{(i)},X_{(i+1)}}(x, y) = \frac{n!}{(i-1)!(n-i-1)!} \left [ F(x) \right ]^{i-1} f(x) \left [ F(y) -  F(x) \right ]^{n-i-1}  f(y) \quad ,x<y
$ 
\end{center}
\end{tcolorbox}

\begin{tcolorbox}[colback=white,colframe=black,width=17cm ,coltitle=white, title = Distribución conjunta de $n$ estadísticos de orden ]
\begin{center}
\[
f_{1,2, \ldots, n}(x_1, x_2, \ldots, x_n) =
\begin{cases}
n! f(x_1)f(x_2) \ldots f(x_n) &, \text{si } -\infty < x_1 \leq x_2 \leq \ldots \leq x_n < \infty;\\
x &, \text{en otro caso}.
\end{cases}
\]
\end{center}
\end{tcolorbox}

\section{Tests}
\subsection*{Rachas}

\begin{center}
    

\begin{tcolorbox}[colback=white,colframe=black]

\begin{multicols}{2} 
\large{
\begin{center}
%\begin{array}{c}
%r_{min} = r_{n_1,n_2,\left ( \frac{\alpha}{2} \right )} \\
%r_{max} = r_{n_1,n_2,\left (1- \frac{\alpha}{2} \right )}
%\end{array}


%r_0 \sim \N(\mu_{r_0},\sigma^2_{r_0}) \\
\large{
Media = $\mu_{r_0} = \frac{2 n_1 n_2}{n_1 + n_2} + 1$ \\
Varianza = $\sigma^2_{r_0}= \frac{2 n_1 n_2 (2 n_1 n_2 - n_1 - n_2)}{(n_1 + n_2)^2 (n_1 + n_2 -1)}$
}
\end{center}}

\end{multicols} 

%\begin{align*}
%   \huge{Z_c = \frac{r_0 - \mu_{r_0}}{\sigma_{r_0}} \sim \N(0,1)}
%\end{align*}

\end{tcolorbox}
\end{center}




\subsection*{Signos}
\begin{multicols}{3} 

\begin{center}
\begin{tcolorbox}[colback=white,colframe=black,width=5cm, title= H_0: M \geq M_o]

\fcolorbox{black}{white}{C > T$^{(-)}$}

  $P_1 = \sum^c_{i = 0}\binom{n}{i}\left ( \frac{1}{2} \right )^n$

\tcblower
\fcolorbox{black}{white}{C < T$^{(-)}$}

  $P_2 = \sum^n_{i = c}\binom{n}{i}\left ( \frac{1}{2} \right )^n$

\end{tcolorbox}
\end{center}




\begin{center}
\begin{tcolorbox}[colback=white,colframe=black,width=5cm, title= H_0: M \leq M_o,coltitle=white]

\fcolorbox{black}{white}{C > T$^{(-)}$}

 $ P_2 = \sum^n_{i = c}\binom{n}{i}\left ( \frac{1}{2} \right )^n$
\tcblower

\fcolorbox{black}{white}{C < T$^{(-)}$}

  $P_1 = \sum^c_{i = 0}\binom{n}{i}\left ( \frac{1}{2} \right )^n$

\end{tcolorbox}
\end{center}






\begin{center}
\begin{tcolorbox}[colback = white,colframe = black, width = 5cm, title = H_0: M \doteq  M_o,coltitle=white]

\fcolorbox{black}{white}{C > T$^{(-)}$}

  $P_2 = \frac{\sum^n_{i = c}\binom{n}{i}\left ( \frac{1}{2} \right )^n}{2}$

\tcblower
\fcolorbox{black}{white}{C < T$^{(-)}$}

$  P_1 = \frac{\sum^c_{i = 0}\binom{n}{i}\left ( \frac{1}{2} \right )^n}{2}$
  
\end{tcolorbox}
\end{center}
\end{multicols} 

\begin{center}
\begin{tcolorbox}[colback=white,colframe=black,width=8cm ,coltitle=white]
$$
Z_c =\frac{(T-0.5)-0.5n}{0.5 \sqrt{n}}
$$
\end{tcolorbox}
\end{center}





\subsection*{Chi Cuadrado}
\begin{center}
\begin{tcolorbox}[colback=white,colframe=black,width=11 cm]
\large{
$$\X^2_{c} = \sum^k_{i=1}\frac{(O_i -E_i)^2}{E_i} \sim \X^2_{(k-1)g.l.$$
}
\end{tcolorbox}
\end{center}







\subsection*{Wilcoxon}
\begin{center}
\begin{tcolorbox}[colback=white,colframe=black,width=15cm ,coltitle=white]

\begin{center}

\Large{
\begin{multicols}{2} 
$\mu_w = \frac{n(n+1)}{4}$\\
$\sigma_w = \sqrt{\frac{n(n+1)(2n+1)}{24}}$\\
%Z_c = \frac{W-\mu_w}{\sigma_w} \sim  \N(0,1)
\end{multicols}
}
    
\end{center}
\end{tcolorbox}
\end{center}





%\end{landscape} 









%\subsection{Kolmogorov-Smirnov}
%\begin{center}
 %   \begin{tcolorbox}[colback=white,colframe=black,width=8cm ,coltitle=white]
%\Large{

%F_n(x)
%\left \{  \begin{array}{rcl}
%&0 \;&  x<x_1\\
%&\frac{1}{n} \;& x_1\leq x \leq x_2\\
%&&\vdots \\
%&\frac{n-1}{1} \;& x_{n-1} \leq x \leq x_{n}\\
%&1 \;& x_n \leq x\\
%\end{array}  
%}
%\end{tcolorbox}
%\end{center}




\begin{center}

\resizebox*{16.5cm}{!}{
\begin{tabular}{|c|c|c|c|c|c}
\hline
\multicolumn{5}{|c|}{\large\textbf{Variables Aleatorias Discretas}} \\ \hline
\textit{\textbf{Distribución}}  & $f(x)$  & $F(x)$  & $E(X)$  & $Var(X)$   \\ \hline 

$Bernoulli\,(p)$    &  \(\displaystyle p^x q^{1-x} \) & \begin{array}{rcl}
0 &\;si\, x<0\\
q &\;si\, 0<x<1\\
1 &\;si\,x>1\
\
\end{array}    &  $p$ &    $pq$    \\ \hline

$Binomial(n,p)$&  \(\displaystyle \binom{n}{x}p^xq^{n-x} \) & \(\displaystyle \sum_{i=1}^x \binom{n}{i}p^iq^{n-i} \)          &     $np$  &   $npq$    \\ \hline

$Poisson(\lambda)$ &  \(\displaystyle \frac{\lambda^x e^{-\lambda}}{x!}  \)    &   \(\displaystyle  e^{-\lambda}\sum_{i=0}^x\frac{\lambda^i}{i!} \)    &  $\lambda$      &  $\lambda$   \\ \hline

$Geométrica(p)$&    \(\displaystyle pq^{x-1} \)    &  \(\displaystyle 1-q^x \)  &   \(\displaystyle \frac{1}{p} \)     &  \(\displaystyle \frac{q}{p^2} \)  \\ \hline

$BinNeg(r,p)$ &   \(\displaystyle  \binom{x-1}{r-1}p^r q^x-r \)    &    \(\displaystyle  1- I_p(x+1,r) \)    &    \(\displaystyle  \frac{r}{p} \)    &  \(\displaystyle  \frac{rq}{p^2} \)   \\ \hline

$HiperGeom(N,M,n)$&  \(\displaystyle \frac{\binom{M}{x}\binom{N-M}{n-x}}{\binom{N}{n}}  \)  &    \(\displaystyle \sum_{i=0}^x\frac{\binom{M}{i}\binom{N-M}{n-i}}{\binom{N}{n}}  \)    &     \(\displaystyle  \frac{nM}{N} \)   &  \(\display \frac{nM}{N}\left (1- \frac{M}{N} \right ) \left (\frac{N-M}{N-1} \right ) \)    \\ 
\hline
\end{tabular}}
    
\end{center}







\begin{center}
\resizebox*{16.5cm}{!}{
\begin{tabular}{|c|c|c|c|c|}
\hline
\multicolumn{5}{|c|}{\large\textbf{Variables Aleatorias Continuas}} \\ \hline
\textit{\textbf{Distribución}} & $f(x)$  & $F(x)$  & $E(X)$  & $Var(X)$  \\ \hline

$Uniforme (a,b) $    &  \(\displaystyle \frac{1}{b-a} \) &   \(\displaystyle \left \{  \begin{array}{rcl}
0 \;si\, x<0\\
q \;si\, x=0\\
1 \;si\, x>0\\
\end{array}  \right.    & \(\displaystyle  \frac{a+b}{2} \)  &     \(\displaystyle \frac{(b-a)^2}{12} \)    \\ \hline

$Exponencial(\lambda) $  &    \(\displaystyle \lambda e^{-\lambda x} \)   &    \(\displaystyle 1 - e^{-\lambda x} \)   &  \(\displaystyle  \frac{1}{\lambda} \)     &     \(\displaystyle \frac{1}{\lambda^2} \)   \\ \hline

$Normal (\mu, \sigma^2)  $     &  \(\displaystyle \frac{1}{\sqrt{2\pi\sigma^2}}e^{-\frac{1}{2}\left ( \frac{x-\mu}{\sigma} \right )^2} \)     &   \(\displaystyle \frac{1}{\sqrt{2\pi}} \int_{-\infty}^x e^{-\frac{t^2}{2}} dt \)    &   \(\displaystyle  \mu \)    &    \(\displaystyle \sigma^2 \)    \\ \hline

$Gamma  (\alpha,\beta) $      &   \(\displaystyle \frac{\beta^\alpha}{\Gamma(\alpha)}x^{\alpha-1}e^{-\beta x} \)    &   \(\displaystyle \int_0^x f(u;\alpha, \beta) \)    & \(\displaystyle \frac{\alpha}{\beta} \)      &   \(\displaystyle  \frac{\alpha}{\beta^2}\)     \\ \hline

$Beta   (\alpha,\beta) $      &   \(\displaystyle \frac{\Gamma(\alpha+\beta)}{\Gamma(\alpha)\Gamma(\beta)}x^{\alpha-1}(1-x)^{\beta -1} \)    &   \(\displaystyle I_x(\alpha, \beta) \)    &   \(\displaystyle \frac{\alpha}{\alpha+\beta} \)    &   \(\displaystyle  \frac{\alpha \beta}{(\alpha+\beta)^2 (\alpha+\beta +1)} \)        \\ \hline

$Weibull  (\alpha,\beta) $    &    \(\displaystyle \frac{\alpha}{\beta}\left ( \frac{x}{\beta} \right )^{\alpha-1} e^{-\left ( \frac{x}{\beta} \right )^{\alpha} } \)   &    \(\displaystyle 1-e^{-\left ( \frac{x}{\beta} \right )^{\alpha} } \)   &    \(\displaystyle \beta \Gamma\left (1+ \frac{1}{\alpha} \right ) \)   &  \(\displaystyle \beta^2\left [ \Gamma\left (1+ \frac{2}{\alpha} \right ) -\Gamma^2\left (1+ \frac{1}{\alpha} \right ) \right ]  \)       \\ \hline

\end{tabular}}
\end{center}


.
\\
.
%%%%%%%%%%%%%%%%%%%%%%%%%%%%%%%%%%%%%%%%%%%%%%%%%%%%%%%%%%%%%

\section*{Para 2 Muestras RELACIONADAS}
\subsection*{Signos}


\begin{center}
\begin{tcolorbox}[colback=white,colframe=black,width=7cm, title= Unilateral]
$$p = P(X \leq x) = \sum_{i=0}^{x} \binom{N}{i} (0.5)^N$$
\end{tcolorbox}
\end{center}


\begin{center}
\begin{tcolorbox}[colback=white,colframe=black,width=7.5cm, title= Bilateral]
$$p = \sum_{i=0}^{x_1} \binom{N}{i} (0.5)^N + \sum_{i=x_2}^{N} \binom{N}{i} (0.5)^N$$
\end{tcolorbox}
\end{center}


%\begin{center}
%\begin{tcolorbox}[colback=white,colframe=black,width=10cm ,coltitle=white]
%\begin{multicols}{2} 
%\large{
%\begin{center}
%\large{
%Media =\mu_{x} = \frac{N}{2}\\
%Varianza =\sigma^2_{x}= \frac{N}{4}
%}
%\end{center}}
%\end{multicols} 
%\end{tcolorbox}
%\end{center}



\begin{center}
\begin{tcolorbox}[colback=white,colframe=black,width=11 cm]
\large{$Z = \frac{(x \pm 0.5) - \frac{N}{2}}{\frac{\sqrt{N}}{2}} $}
donde
\left\{\begin{matrix}
$x + 0.5 \quad para & x < \frac{N}{2}$\\ 
$x - 0.5 \quad para & x > \frac{N}{2}$
\end{matrix}\right. 
\end{tcolorbox}
\end{center}



\subsection*{McNemar}
\begin{center}
\begin{tcolorbox}[colback=white,colframe=black,width=8 cm]
\large{
$$T_{1} = \frac{(\left | B-C \right | - 1)^2}{B+C} $$
}
\end{tcolorbox}
\end{center}

%\sim \X^2_{(1)g.l.



\subsection*{Rangos Señalados y Pares Igualados de Wilcoxon}

\begin{center}
\begin{tcolorbox}[colback=white,colframe=black,width=14cm ,coltitle=white]
\begin{multicols}{2} 
\large{
\begin{center}
\large{
$\mu_{x} = \frac{N(N+1)}{4}$\\
$\sigma^2_{x}= \frac{N(N+1)(2N+1)}{24}$
}
\end{center}}
\end{multicols} 
\end{tcolorbox}
\end{center}




%%%%%%%%%%%%%%%%%%%%%%%%%%%%%%%%%%%%%%%%%%

\section*{Para 2 Muestras INDEPENDIENTES}
\subsection*{La Mediana}
\begin{center}
\begin{tcolorbox}[colback=white,colframe=black,width=12 cm]
\large{
$$\X^2 = \frac{n(\left | AD-BC \right | - \frac{n}{2})^2}{(A+B)(C+D)(A+C)(B+D)} $$
}
\end{tcolorbox}
\end{center}
%\sim \X^2_{(1)g.l.} = \X^2_{tabla,(1)g.l.}




\subsection*{U de Mann-Whitney}

\begin{center}
\begin{tcolorbox}[colback=white,colframe=black,width=15cm ,coltitle=white]
\begin{multicols}{2} 
\large{
\begin{center}
\large{
$U_{1} = n_1 n_2 +\frac{n_1(n_1+1)}{2} - \sum_{i=1}^{n_1} R_i$ \\
$U_{2}=  n_1 n_2 +\frac{n_2(n_2+1)}{2} - \sum_{i=1}^{n_2} R_i$
}
\end{center}}
\end{multicols} 
\end{tcolorbox}
\end{center}



\begin{center}
\begin{tcolorbox}[colback=white,colframe=black,width=14cm ,coltitle=white]
\begin{multicols}{2} 
\large{
\begin{center}
\large{
$\mu_{U} = \frac{n_1 n_2}{2}$\\
$\sigma^2_{U}= \frac{n_1 n_2(n_1 + n_2 + 1)}{12}$
}
\end{center}}
\end{multicols} 
\end{tcolorbox}
\end{center}




\subsection*{\mbox{\boldmath $X^2$}}
\begin{center}
\begin{tcolorbox}[colback=white,colframe=black,width=8.5cm ,coltitle=white]
$$
X^2_{c} = \sum^{r}_{i=1} \sum^{2}_{j=1} \frac{(O_{ij} - e_{ij})^2}{e_{ij}}$$
\end{tcolorbox}
\end{center}
 %\sim \X^2_{\alpha,(r-1)\times (1)g.l.} 

\begin{center}
\begin{tcolorbox}[colback=white,colframe=black,width=8cm ,coltitle=white]
$$
X^2_{c} =\frac{n(\left | AD-BC \right | - \frac{n}{2})^2}{n_{\cdot 1}n_{\cdot 2}n_{1 \cdot}n_{2 \cdot}}$$
\end{tcolorbox}
\end{center}

% \sim \X^2_{\alpha,(1)g.l.} 


\subsection*{La Probabilidad exacta de FISHER}
\begin{center}
\begin{tcolorbox}[colback=white,colframe=black,width=8cm ,coltitle=white]
$$
P_{a'} =\frac{(a+b)!(c+d)!(a+c)!(b+d)!}{n!\; a'!\;b'!\;c'!\;d'!}
$$
\end{tcolorbox}
\end{center}


\subsection*{Kolmogorov-Smirnov}

\begin{center}
\begin{tcolorbox}[colback=white,colframe=black,width=6 cm]
\large{
$$\X^2_c = 4D^2_c\frac{n_1 n_2}{n_1 + n_2} $$}
\end{tcolorbox}
\end{center}
%\sim \X^2_{(2 g.l.)} 



%Esto es para empezar el texto en la siguiente pagina
\newpage
%Si quieres agregar una pagina en blanco seria con estos comandos
%\afterpage{\null\newpage}
%\newpage

\section*{Para K Muestras RELACIONADAS}
\subsection*{Friedman}
\begin{center}
\begin{tcolorbox}[colback=white,colframe=black,width=8cm ,coltitle=white]
$$
X^2_{c} =\frac{12}{nk(k+1)} \sum^{k}_{j=1} R^2_j - 3n(k+1)$$
\end{tcolorbox}
\end{center}

\subsection*{Q de Cochran}
\begin{center}
\begin{tcolorbox}[colback=white,colframe=black,width=10cm ,coltitle=white, valign = top] 
\begin{equation*}Q_{c} =\frac{ (k-1)\left[k\sum\limits^{k}_{j=1} G^2_j - \left(\sum\limits^{k}_{j=1} G_j \right)^2\right] }{k \sum\limits^{n}_{i=1} L_i - \sum\limits^{n}_{i=1} L_i^2} 
\end{equation*}
\end{tcolorbox}
\end{center}


\section*{Para K Muestras INDEPENDIENTES}
\subsection*{$X^2$}
\begin{center}
\begin{tcolorbox}[colback=white,colframe=black,width=8.5cm ,coltitle=white]
$$
X^2_{c} = \sum^{r}_{i=1} \sum^{2}_{j=1} \frac{(O_{ij} - e_{ij})^2}{e_{ij}}$$
\end{tcolorbox}
\end{center}

\subsection*{Kruskall-Wallis}
\begin{center}
\begin{tcolorbox}[colback=white,colframe=black,width=8.5cm ,coltitle=white]
$$
H =\frac{12}{N(N+1)} \sum^{k}_{j=1} \frac{R^2_j}{n_j} - 3n(k+1)$$
\end{tcolorbox}
\end{center}


\begin{center}
\begin{tcolorbox}[colback=white,colframe=black,width=8.5cm ,coltitle=white]
$$T_i = t_i^3 - t_i$$
$$H_{corregido} =\frac{H}{\left[ 1- \frac{\sum\limits^{t}_{i=1} T_i}{N^3-N}\right]}$$
\end{tcolorbox}
\end{center}




%Para poner una ecuacion en la tabla
%\(\displaystyle \sum_{n=1}^2 C_n \)



%PAra colocar muchas ecauciones en una celda de la tabla
%\multicolumn{1}{|p{3cm}|}{
%         \begin{gather*}
%           y = x^2 \\
%            z = \log(y)\\
%           w = 3z + 2x
%         \end{gather*}
%       }


\end{document}